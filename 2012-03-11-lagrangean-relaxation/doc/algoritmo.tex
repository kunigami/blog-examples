\documentclass[12pt]{article}

\usepackage{graphicx}
\usepackage[brazilian]{babel}
\usepackage[utf8]{inputenc}

\usepackage{amsmath}
\usepackage{amssymb}
\usepackage{amsfonts}
\usepackage{amsthm}

\begin{document}

\emph{Método do Subgradiente}($\pi, T_\pi, N, u_0$)

\begin{enumerate}
 \setlength{\itemsep}{0.7pt}
\item $\bar x^*$ //\emph{Melhor solução primal}
\item Enquanto condição de parada não for satisfeita, faça:
\item \qquad $x^* \leftarrow$ \emph{Dual\_lagrangeano}()
\item \qquad Se $f'(x^*) \ge z_{UB}$, faça:
\item \qquad \qquad $z_{LB} \leftarrow f'(x^*)$;
\item \qquad \qquad $n_{stuck} \leftarrow 0$;
\item \qquad Caso contrário faça:
\item \qquad \qquad $n_{stuck} \leftarrow n_{stuck} + 1$;
\item \qquad $\bar x \leftarrow$ \emph{Heurística\_Lagrangeana()};
  //\emph{Solução primal viável}
\item \qquad Se o valor de $f(\bar x) \le z_{LB}$, faça: 
\item \qquad \qquad $z_{LB} \leftarrow f(\bar x)$;
\item \qquad \qquad $\bar x^* \leftarrow \bar x$;
\item \qquad Se $z_{LB} = z_{UB}$ //\emph{Encontramos a solução ótima}: 
\item \qquad \qquad PARE;
\item \qquad $u_l^{k + 1} \leftarrow$;
  \emph{Atualiza\_multiplicadores}($\pi, z_{LB}, z_{UB}, u_l^{k}, x^*$);
\item \qquad Se $n_{stuck} = N$, faça:
\item \qquad \qquad $n_{stuck} \leftarrow 0$;
\item \qquad \qquad $\pi \leftarrow \pi / 2.0$;
\item \qquad \qquad Se $\pi \le T_\pi$:
\item \qquad \qquad \qquad PARE;
\item \qquad $k \leftarrow k + 1$;
\item Retorne $\bar x^*$;

\end{enumerate}

\clearpage

\emph{Atualiza\_multiplicadores}($\pi, z_{LB}, z_{UB}, u_l^{k}, x^*$) 

\begin{enumerate}
\item $G_\ell = b_\ell - \sum_{j = 1}^m a_{\ell j} x^*_{\ell j} \qquad \forall \ell \in R$
\item $T = \dfrac{\pi(1.05 \cdot z_{UB} - z_{LB})}{\sum_{\ell \in R} G_\ell^2}$
\item $u^{k + 1}_\ell = \max \{ 0, u^{k}_\ell + TG_\ell \}  \qquad
  \forall \ell \in R$
\item Retorne $u^{k + 1}$
\end{enumerate}


\bibliographystyle{plain}
\bibliography{refer} 
 
\end{document}

